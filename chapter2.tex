\chapter{The PerLa System}

PerLa is a software infrastructure for data management and integration in
Pervasive Information Systems. Its development began in 2005 at Politecnico di
Milano, with a thesis by Marco Marelli and Marco Fortunato entitled ``\textit{A
Declarative Language for Pervasive Systems}'' \cite{mm_thesis}. With this
document the two authors laid the foundations of a completely declarative,
SQL-like language that could be used to gather information from Pervasive
Systems and Wireless Sensing Networks. Though their work primarily focussed on
defining the syntax and semantics of the PerLa language, Marelli and Fortunato
went on to propose a reference software architecture that could support the
execution of PerLa data collection queries.

In their first design they envisioned the possibility of creating a
\textit{Device Access Layer} whose goal was to conceal all the idiosyncratic
features of a Pervasive System, and provide a homogeneous data access interface
that could be used as a thought device to support the first development stages
of PerLa. The principal element of this earliest software architecture was the
\textit{Logical Object}, a virtualization module that provides a uniform API
for accessing the functionalities of a single device in the sensing network.
This germinal architecture evolved during the course of the following years
into what would later be known as \textit{PerLa Middleware}. 

The PerLa Middleware~\cite{tse_perla} was primarily concerned with providing an
actual implementation of the Logical Object abstraction, a goal that was
achieved with the creation of the Functionality Proxy Component (\texttt{FPC}).
The \texttt{FPC} is a Java entity that reifies all concepts embodied by the
Logical Object, with particular emphasis on the ability to abstract the
peculiar features of a single sensing device through a common and uniform
programming interface. The PerLa Middleware, however, was more than a simple
implementation of the Logical Object, as it provided a \textit{Plug \& Play}
system for the autonomous creation of \texttt{FPC} objects and an initial
implementation of the PerLa Language Executor component.

The development of the PerLa Middleware has been a collaborative effort that
involved multiple students and several years of work. The product that was
created is thus the sum of all contributions made by different people that, at
one time or another, put their minds (and hearts) at the design and
implementation of the system. While this development methodology allowed PerLa
to thrive and mature rapidly, since many intellects had the possibility to
contribute with their innovative ideas, it also meant that its growth has been
dishomogeneous and, at times, chaotic. It is under these premises that, in
early 2014, the Middleware underwent a complete redesign, whose primary
intentions were to consolidate the main programmatic API (Application
Programming Interface), improve performances, and further advance some of the
defining characteristics of the system. This thesis is a description of this
recent endeavor, and will continue by describing the past, present and
foreseeable future of the PerLa System.

The remaining sections of this chapter provide an account of PerLa prior to the
current redesign, starting with the previous Middleware architecture --- also
known as Classic PerLa Middleware --- and ending with a short digression
towards Context management in Pervasive Systems. This same chapter will
describe the main features of the PerLa Query Language, which as of today still
is the interface of choice for using PerLa.
Chapters~\ref{cha:middleware_overview} and~\ref{cha:components} contain an
in-depth description of the New Middleware architecture, which concerns any
developer interested in developing new PerLa Plugins or connecting new types of
sensing devices. Finally, chapter~\ref{cha:conclusions} wraps up the work
illustrated in this thesis, and provides an outlook on the prospects for
PerLa's future.


\section{The Classic Middleware Architecture}

\begin{itemize}

    \item data-centric view of the pervasive systems;

    \item homogeneous high level interface to heterogeneous
    devices;

    \item support for highly dynamic networks (e.g., wireless
    sensor networks);

    \item minimal coding effort for new device addition.

\end{itemize}

The result of this approach is the possibility to access all data generated by
the sensing network via an SQL-like query language, called the PerLa Language,
that allows end users and high level applications to gather and process
information without any knowledge of the underlying pervasive system. Every
detail needed to interact with the network nodes, such as hardware and software
idiosyncrasies, communication paradigms and protocols, computational
capabilities, etc., is completely masked by the PerLa Middleware.

The main software module of the low level support layer is the Functionality
Proxy Component (FPC). As suggested by its name, the FPC’s primary task is to
act as a proxy among sensing nodes and the rest of the PerLa System. Owing to
the FPC’s homogeneous access interface, components of the High level support
layer can communicate with the physical devices ignoring their functional
behaviour. As a matter of fact, due to the abstraction provided by the
Functionality Proxy Component, all nodes of the pervasive system appear to
implement a common interface.

The information generated by the physical sensing nodes are abstracted as FPC
Attributes. Attributes can be used to retrieve the node state (e.g., battery
status, memory occupation, etc.), to access sampled data (e.g., temperature,
pressure, etc.), or to change some parame- ters on the device (sampling
frequency, node parameters, etc.). 

PerLa supports the introduction of new
kinds of nodes at run-time through a Plug \& Play device addition mechanism.
FPCs are created by means of the FPC Factory, a software module responsible for
PerLa’s Plug \& Play system. New nodes can join a network managed by PerLa
simply by forwarding their XML device descriptor to the FPC Factory, which
creates a Functionality Proxy Component suitable to handling the node.

The Query Parser and the Query Analyser compose the user interface of the
entire PerLa System.

\section{The PerLa Query Language}

The PerLa Query Language is a declarative, SQL-like language for interacting
with Pervasive Systems. A sensing network managed by PerLa is abstracted as a
large table in a streaming database, whose columns correspond to specific data
elements that can be retrieved from the devices connected to the Middleware.
This generalization allows final users to glean information out of a Pervasive
System without dealing with all the complications that stem from managing the
quirks of every single sensing node, as the intricate mesh of available data
sources is completely hidden by the database abstraction.

The PerLa Query Language is designed to be simple and easy to use. Its core
syntax is compact and reminiscent of other well-known database-oriented
languages as SQL. It provides a uniform interrogation mechanism that enables
the collection of data elements regardless of their origin. Information may be
sampled from a physical phenomena, read from the memory of an endpoint device
or estracted from a web service; whatever the source, the PerLa query pattern
is always the same.

PerLa queries allow the user to determine the exact behaviour of a data source
using a consice but powerful set of clauses. The reminder of this section will
provide an overview of the main syntactic and semantic features of the PerLa
Query Language, along with two examples excerpted from real-world use cases.

\subsection{The Data Management section}

Introduced by the \texttt{SELECT} clause, this section of the PerLa Query
Language should immediately result familiar to every person acquainted with the
SQL language. This clause achieves two purposes: first, it defines which data
elements (specifically, which data \texttt{Attributes}) are to be collected
from the Pervasive System; second, it indicates the operations and computations
that must be performed on the information extracted from the Pervasive System.

The need to manage a theoretically infinite stream of data elements coming from
the sensing network required the development of a custom syntax for aggregate
operations. Differently from standard SQL aggregates, which always operate on a
finite set of elements whose size is well-known at runtime, PerLa aggregates
must cope with an ever-flowing stream of records, and thus require users to
specify the scope of their intended computations. This is achieved through a
duration expression, an additional mandatory parameter that complements the
aggregation expression by limiting the number of records to be processed to a
limited amount. Duration expressions can be specified using two different
methods: a time-based syntax, that allows users to define the aggregation scope
in terms of time windows (\lstinline!SELECT AVG(TEMP, 10 SECONDS)!), and a
record-based syntax, that clearly indicates the number of records to be used
for the computation (\lstinline!SELECT AVG(TEMP, 30 SAMPLES)!).

\subsection{The Sampling section}

The Sampling section can be used to specify how and when the data elements
requested with the \texttt{SELECT} statement are to be extracted from the
network nodes. There are two different operating modes, both introduced by the
\texttt{SAMPLING} clause. \textit{Time-based} sampling can be used to collect
data at periodic intervals. The sampling frequency is specified by means of an
\texttt{IF-EVERY} syntax that enables users to specify different sampling
periods, along with the conditions for their activations. On the other hand,
the \textit{Event-based} sampling mode allows the acquisition of a data sample
each time the desired event is fired.

~\\
\begin{lstlisting}[caption={An example of time-based sampling, which shows how
the sampling frequency can be increased as the monitored phenomenon evolves.}]
SAMPLING
    IF temperature < 50 EVERY 10 MINUTES
    ELSE IF TEMPERATURE >= 50 EVERY 1 MINUTES
\end{lstlisting}

\subsection{The Conditional Execution section}

Introduced by the \texttt{EXECUTE IF} clause, this query section contains a
boolean expression that every sensing device must satisfy in order to be
considered as a candidate data source, and it's often employed when the user
requires its query to be executed on nodes with well-defined capabilities. This
section is not mandatory, and its omission implies that the PerLa query must be
executed on every device of the sensing network. An \texttt{EXECUTE IF}
statement can be optionally complemented by a \texttt{REFRESH} clause, which
specifies how often the execution condition is re-evaluated to update the list
of nodes involved in the evaluation of a query. 

\subsection{The Termination Condition section}

An optional clause that can be used to terminate the execution of a query, both
in terms of time (\lstinline!TERMINATE AFTER 1 DAY!) or number of selections
performed (\lstinline!TERMINATE AFTER 10 SELECTIONS!). This behaviour is useful
when perform a one-shot query, or when the monitoring period is known a priori.

\subsection{Query examples}

The following query initiates a temperature sampling operation on all
temperature sensors located in room number three. New data readings are
collected by the minute, as specified by the \texttt{SAMPLING} clause; however,
new output records are created every 5 minutes, as indicated in the
\texttt{EVERY} statement that guards the data management section. Finally, each
record produced by this query contains the maximum temperature value collected
in the previous 10 minutes of sampling.

\begin{lstlisting}
CREATE OUTPUT STREAM Table (Temperature FLOAT) AS:
EVERY 5 MINUTES
SELECT MAX(temp, 10 MINUTES)
SAMPLING
  EVERY 1 MINUTES
EXECUTE IF EXISTS(temp) AND EXISTS(room) AND room = 3
\end{lstlisting}


This second example illustrates how the PerLa Language can be used to collect
information in response to an event. First of all, this is a one-shot query, as
it terminates as soon as the first record is produced. The single output record
contains the number of times the RFID with identifier \texttt{0xDF445A} was
scanned in the last 10 minutes.

\begin{lstlisting}
CREATE OUTPUT STREAM Table (rfid STRING, counter INTEGER) AS:
EVERY 10 MINUTES
SELECT lastReaderId, COUNT(*, 10 MINUTES)
SAMPLING
  ON EVENT lastReaderChanged
EXECUTE IF ID=[0xDF445A]
TERMINATE AFTER 1 SELECTIONS
\end{lstlisting}


\section{Context management}

