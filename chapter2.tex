\chapter{The PerLa System}

PerLa is a software infrastructure for data management and integration in
Pervasive Information Systems. Its development began in 2005 at Politecnico di
Milano, with a thesis by Marco Marelli and Marco Fortunato entitled ``\textit{A
Declarative Language for Pervasive Systems}'' \cite{mm_thesis}. With this
document the two authors laid the foundations of a completely declarative,
SQL-like language that could be used to gather information from Pervasive
Systems and Wireless Sensing Networks. Though their work primarily focussed on
defining the syntax and semantics of the PerLa language, Marelli and Fortunato
went on to propose a reference software architecture that could support the
execution of PerLa data collection queries.

In their first design they envisioned the possibility of creating a
\textit{Device Access Layer} whose goal was to conceal all the idiosyncratic
features of a Pervasive System, and provide a homogeneous data access interface
that could be used as a thought device to support the first development stages
of PerLa. The principal element of this earliest software architecture was the
\textit{Logical Object}, a virtualization module that provides a uniform API
for accessing the functionalities of a single device in the sensing network.
This germinal architecture evolved during the course of the following years
into what would later be known as \textit{PerLa Middleware}. 

The PerLa Middleware~\cite{tse_perla} was primarily concerned with providing an
actual implementation of the Logical Object abstraction, a goal that was
achieved with the creation of the Functionality Proxy Component (\texttt{FPC}).
The \texttt{FPC} is a Java entity that reifies all concepts embodied by the
Logical Object, with particular emphasis on the ability to abstract the
peculiar features of a single sensing device through a common and uniform
programming interface. The PerLa Middleware, however, was more than a simple
implementation of the Logical Object, as it provided a \textit{Plug \& Play}
system for the autonomous creation of \texttt{FPC} objects and an initial
implementation of the PerLa Language Executor component.

The development of the PerLa Middleware has been a collaborative effort that
involved multiple students and several years of work. The product that was
created is thus the sum of all contributions made by different people that, at
one time or another, put their minds (and hearts) at the design and
implementation of the system. While this development methodology allowed PerLa
to thrive and mature rapidly, since many intellects had the possibility to
contribute with their innovative ideas, it also meant that its growth has been
dishomogeneous and, at times, chaotic. It is under these premises that, in
early 2014, the Middleware underwent a complete redesign, whose primary
intentions were to consolidate the main programmatic API (Application
Programming Interface), improve performances, and further advance some of the
defining characteristics of the system. This thesis is a description of this
recent endeavor, and will continue by describing the past, present and
foreseeable future of the PerLa System.

The remaining sections of this chapter provide an account of PerLa prior to the
current redesign, starting with the previous Middleware architecture --- also
known as Classic PerLa Middleware --- and ending with a short digression
towards Context management in Pervasive Systems. This same chapter will
describe the main features of the PerLa Query Language, which as of today still
is the interface of choice for using PerLa.
Chapters~\ref{cha:middleware_overview} and~\ref{cha:components} contain an
in-depth description of the New Middleware architecture, which concerns any
developer interested in developing new PerLa Plugins or connecting new types of
sensing devices. Finally, chapter~\ref{cha:conclusions} wraps up the work
illustrated in this thesis, and provides an outlook on the prospects for
PerLa's future.


\section{The Classic Middleware Architecture}

\begin{itemize}

    \item data-centric view of the pervasive systems;

    \item homogeneous high level interface to heterogeneous
    devices;

    \item support for highly dynamic networks (e.g., wireless
    sensor networks);

    \item minimal coding effort for new device addition.

\end{itemize}

The result of this approach is the possibility to access all data generated by
the sensing network via an SQL-like query language, called the PerLa Language,
that allows end users and high level applications to gather and process
information without any knowledge of the underlying pervasive system. Every
detail needed to interact with the network nodes, such as hardware and software
idiosyncrasies, communication paradigms and protocols, computational
capabilities, etc., is completely masked by the PerLa Middleware.

The main software module of the low level support layer is the Functionality
Proxy Component (FPC). As suggested by its name, the FPC’s primary task is to
act as a proxy among sensing nodes and the rest of the PerLa System. Owing to
the FPC’s homogeneous access interface, components of the High level support
layer can communicate with the physical devices ignoring their functional
behaviour. As a matter of fact, due to the abstraction provided by the
Functionality Proxy Component, all nodes of the pervasive system appear to
implement a common interface.

The information generated by the physical sensing nodes are abstracted as FPC
Attributes. Attributes can be used to retrieve the node state (e.g., battery
status, memory occupation, etc.), to access sampled data (e.g., temperature,
pressure, etc.), or to change some parame- ters on the device (sampling
frequency, node parameters, etc.). 

PerLa supports the introduction of new
kinds of nodes at run-time through a Plug \& Play device addition mechanism.
FPCs are created by means of the FPC Factory, a software module responsible for
PerLa’s Plug \& Play system. New nodes can join a network managed by PerLa
simply by forwarding their XML device descriptor to the FPC Factory, which
creates a Functionality Proxy Component suitable to handling the node.

The Query Parser and the Query Analyser compose the user interface of the
entire PerLa System.

\section{The PerLa Query Language}

The PerLa Query Language is a declarative, SQL-like language for interacting
with Pervasive Systems. A sensing network managed by PerLa is abstracted as a
large table in a streaming database, whose columns correspond to specific data
elements that can be retrieved from the devices connected to the Middleware.
This generalization allows final users to glean information out of a Pervasive
System without dealing with all the complications that stem from managing the
quirks of every single sensing node, as the intricate mesh of available data
sources is completely hidden by the database abstraction.

The PerLa Query Language is designed to be simple and easy to use. Its core
syntax is compact and reminiscent of other well-known database-oriented
languages as SQL. It provides a uniform interrogation mechanism that enables
the collection of data elements regardless of their origin. Information may be
sampled from a physical phenomena, read from the memory of an endpoint device
or estracted from a web service; whatever the source, the PerLa query pattern
is always the same.

PerLa queries allow the user to determine the exact behaviour of a data source
using a consice but powerful set of clauses. The reminder of this section will
provide an overview of the main syntactic and semantic features of the PerLa
Query Language, along with two examples excerpted from real-world use cases.

\subsection{The \textbf{SAMPLING} section}





\section{Context management}

