\chapter{The PerLa System}

PerLa is a software infrastructure for data management and integration in
Pervasive Information Systems. Its development began in 2005 at Politecnico di
Milano, with a thesis by Marco Marelli and Marco Fortunato entitled ``\textit{A
Declarative Language for Pervasive Systems}'' \cite{mm_thesis}. With this
document the two authors laid the foundations for the implementation of a
completely declarative, SQL-like language that could be used to gather
information from Pervasive Systems and Wireless Sensing Networks. Though their
work primarily focussed on defining the syntax and semantics of the PerLa
language, Marelli and Fortunato went on to propose a reference software
architecture that could support the execution of PerLa data collection queries.

In their first design they envisioned the possibility of creating a
\textit{Device Access Layer}, whose goal was to conceal all the hidiosincratic
features of a Pervasive System and provide a homogeneous data access interface
that could be used to develop the PerLa language. The principal element of this
earliest software architecture was the \textit{Logical Object}, a
virtualization module that provides a uniform API for accessing the
functionalities of a single device in the sensing network.

This initial design later evolved with the 

quickly evolved to include a middleware and a 

The development of the PerLa System focused on the design and implementation of
the following features:

\begin{itemize}

    \item data-centric view of the pervasive systems;

    \item homogeneous high level interface to heterogeneous
    devices;

    \item support for highly dynamic networks (e.g., wireless
    sensor networks);

    \item minimal coding effort for new device addition.

\end{itemize}

The result of this approach is the possibility to access all data generated by
the sensing network via an SQL-like query language, called the PerLa Language,
that allows end users and high level applications to gather and process
information without any knowledge of the underlying pervasive system. Every
detail needed to interact with the network nodes, such as hardware and software
idiosyncrasies, communication paradigms and protocols, computational
capabilities, etc., is completely masked by the PerLa Middleware.

The main software module of the low level support layer is the Functionality
Proxy Component (FPC). As suggested by its name, the FPC’s primary task is to
act as a proxy among sensing nodes and the rest of the PerLa System. Owing to
the FPC’s homogeneous access interface, components of the High level support
layer can communicate with the physical devices ignoring their functional
behaviour. As a matter of fact, due to the abstraction provided by the
Functionality Proxy Component, all nodes of the pervasive system appear to
implement a common interface.

The information generated by the physical sensing nodes are abstracted as FPC
Attributes. Attributes can be used to retrieve the node state (e.g., battery
status, memory occupation, etc.), to access sampled data (e.g., temperature,
pressure, etc.), or to change some parame- ters on the device (sampling
frequency, node parameters, etc.). 

PerLa supports the introduction of new
kinds of nodes at run-time through a Plug \& Play device addition mechanism.
FPCs are created by means of the FPC Factory, a software module responsible for
PerLa’s Plug \& Play system. New nodes can join a network managed by PerLa
simply by forwarding their XML device descriptor to the FPC Factory, which
creates a Functionality Proxy Component suitable to handling the node.

The Query Parser and the Query Analyser compose the user interface of the
entire PerLa System.

\subsection{The PerLa query language}

\section{Context management}

