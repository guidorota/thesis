\chapter{Conclusions}
\label{cha:conclusions}

This thesis described the design and implementation of an asynchronous data
access middleware for Pervasive Systems. As shown in previous chapters, the
implementation of this middleware began with an analysis of the existing data
access layer employed to support the execution of PerLa Queries, aimed at
identifying its weaknesses and strength; a series of requirements was 


\section{Future work}

\subsection{Implementation of new plugins}

The New PerLa Middleware is designed to be expanded through the addition of new
modules, and it should come as no surprise that one of its most natural
evolution paths consists in fact in the development of new Plugins. As
described in chapter~\ref{cha:components}, there are two main types of modules
that may be added with the PerLa Plugin System: \texttt{Channels} and
\texttt{Mappers}.

The possibility to add new \texttt{Channel} implementations is a distinguishing
feature of the New Middleware that can be exploited to enlarge the type of
supported endpoint devices. Future \texttt{Channel}s, should be implemented to


A natural evolution path for the New PerLa Middleware consist in the expansion
of the core functionalities through the implementation of new Plugins. This

\subsection{Alternative Device Descriptor forms}

Chapter~\ref{cha:components} described the 

\subsection{Distributed PerLa}
