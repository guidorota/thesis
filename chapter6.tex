\chapter{In-depth component description}

\section{Communicating with Channels}

\texttt{Channel} is an interface for performing I/O operations. It represents the principal abstraction used by the middleware to communicate with hardware devices and external software services. Thanks to their generic interface, \texttt{Channel}s can be used for a wide variety of I/O taks, including but not limited to networking, file management, and automatic data generation.

All \texttt{Channel}s are created open and ready to be used. They may be optionally closed to relinquish resources by invoking the \texttt{close()} method. Once closed, a \texttt{Channel} cannot be opened again, and every subsequent attempt to perform an IO operation will fail causing a \texttt{ChannelException} to be thrown. The current state of a \texttt{Channel} can be probed through its \texttt{isClosed()} method.

New I/O operations are started using the \texttt{submit()} method. As can be seen by reading the signature available in listing~\ref{lst:channel.submit}, this method is a direct implementation of the asynchronous invocation paradigm introduced in section~\ref{sec:newmiddleware.async}. Calls to \texttt{submit()} are non-blocking, and require the caller to specify an \texttt{IOHandler} through which data and errors will be notified once the requested operation is fulfilled. 

Explain the methods in the IOHandler interface


Ongoing I/O operations can be handled through the \texttt{IOTask} object, which is returned upon submitting a new request

\lstset{language=Java}
\begin{lstlisting}[float,floatplacement=H,caption=The Channel.submit() method signature,label={lst:channel.submit}]
public IOTask submit(IORequest request, IOHandler handler)
	throws ChannelException;
\end{lstlisting}






\lstset{language=Java}
\begin{lstlisting}[float,floatplacement=H,caption=The Channel interface,label={lst:channel}]
public interface Channel {
	public String getId();
	
	public IOTask submit(IORequest request, IOHandler handler)
			throws ChannelException;
	
	public void setAsyncIOHandler(IOHandler handler)
			throws IllegalStateException;
			
	public boolean isClosed();
	
	public void close();		
}
\end{lstlisting}









There is no limit on the number of channels used by an FPC. Complex behaviours can be implemented using different channels

Interface description: methods, IORequest, IOHandler IOTask and Payload.
ByteArrayPayload description

The new Channel structure is monolithic, in that it contains all network layers required for the communication. This is in contrast with the previous middleware architecture, where communication layers where diveded between the legacy Channel implementation and the FPC. This new structure allows better reuse of off-the-shelf libraries.

ChannelFactory interface, methods and generic ChannelDescriptor structure

Basic XML 

\subsection{IORequest management}

Customizing request with parameters

IORequestBuilder and IORequestBuilderFactory

\subsection{Handling asynchronous data transmissions}

setAsyncIOHandler, how does it work and why do we need it

\subsection{Implementations: HTTPChannel and SimulatorChannel}

Full examples of actual implementations, complete with XML descriptor snippets


\section{Encoding and decoding information}

\subsection{The Message and Mapper interfaces}

\subsection{Handling composite data structures}

\subsection{Managing multiple message types}

\subsection{Implementations: JSONMapper and URLEncodedMapper}


\section{Data management: Scripts}

\subsection{From Messages to Records}

\subsection{Available instructions}

\subsection{Engine architecture and execution model}

\subsection{Script examples}


\section{Putting it all together: the FPC}

\subsection{Data access interface}

\subsection{Controlling the remote device}

\subsection{Scheduling mechanism}


\section{Device Descriptor and FPC Factory}

\subsection{The XML Device Descriptor}

\subsection{FPC Factory}

\subsection{Registry}

\subsection{Complete XML Device Descriptor examples}