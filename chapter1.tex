\chapter{Introduction}

Computing devices permeate every aspect of our life. PCs, tablets, smartphones,
identification badges, credit cards, smart-watches, wearable gadgets, traffic
cameras, digital fitness bands, and personal medical devices like pacemakers or
insulin injectors are only a few of the tools that we use every day, more or
less consciously, to produce and consume information.

As the number of devices and services that surround ourselves increases, so
does the level of mutual cooperation that we expect from them. We know that our
smartphone will automatically show us the weather for our current location,
sometimes even for our hometown when we are away (How does it know where I
live? Did I ever tell it?). Navigation apps guide us through different
itineraries at different times of the day, depending on current and expected
traffic conditions. Outbreaks of the most common viral diseases can be traced
and monitored by analysing what people are searching on the web and correlating
that data with other sources like hospital records.

We have grown so accustomed to the tight level of integration between different
services and information sources, that behaviours similar to those presented
above are nowadays expected. User requirements have heightened, and products
can fail to gain traction if they don’t exploit the data that is available
around us in new and innovative ways; different computing devices and services
must discover themselves and make mutual use of the information that they
produce or consume, meshing together in what is called a Pervasive System.

Firstly envisioned by Mark Weiser [], Pervasive Systems are connected networks
of independent and heterogeneous devices, whose ultimate goal is to assist
people in a way that is effectively invisible to the final user. They are the
result of a post-pc era, where scores of computing gadgets are disseminated in
our surroundings, enhancing our capabilities to sense the world and providing
ubiquitous access to information.

From a software and hardware perspective, a Pervasive Systems is a rich and
varied environment, hosting myriads of different network protocols, data
formats, and incompatible cpu architectures. Tapping the ever increasing stream
of data produced in such a heterogeneous context, and using it to build
advanced and connected products, can easily become a daunting task. Since
Weiser’s seminal paper, several endeavours have explored different techniques
for simplifying the task of building and designing Pervasive Systems, many of
    which stemmed from the research on Wireless Sensor Networks (WSN).

\section{Wireless Sensor Networks and beyond} As suggested by the name,
Wireless Sensor Networks (WSNs) are networks of wirelessly connected devices
called nodes or motes, that are able to measure or detect physical properties
from their surrounding environment. Although the original acronym only
references the sensory features of such systems, current usage of the term WSN
is commonly extended to include devices with actuation capabilities.

Wireless Sensor Networks provide a low-cost and effective solution for
monitoring physical phenomena: several inexpensive nodes may be scattered
around the area of observation, without requiring an explicit configuration or
a wired communication infrastructure. Data is autonomously routed from the
point of origin to the interested consumers, where it is usually aggregated,
analysed and presented to the user. Flexibility and ease of deployment make
WSNs the ideal candidate for a plethora of applications, covering home
automation, theft prevention systems, healthcare, control of environmental
hazards and monitoring of production lines.

The rapid increase in popularity of WSNs, coupled with the intrinsic
difficulties of working with several computing 



\chapter{Data management in Pervasive Systems}

Stato dell'arte, riprenderlo dalla TSE
