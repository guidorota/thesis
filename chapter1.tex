\chapter{Introduction}

Computing devices permeate every aspect of our life. PCs, tablets, smartphones,
identification badges, credit cards, smart-watches, wearable gadgets, traffic
cameras, digital fitness bands, and personal medical devices like pacemakers or
insulin injectors are only a few of the tools that we use every day, more or
less consciously, to produce and consume information.

As the number of devices and services that surround ourselves increases, so
does the level of mutual cooperation that we expect from them. We know that our
smartphone will automatically show us the weather for our current location,
sometimes even for our hometown when we are away (How does it know where I
live? Did I ever tell it?). Navigation apps guide us through different
itineraries at different times of the day, depending on current and expected
traffic conditions. Outbreaks of the most common viral diseases can be traced
and monitored by analysing what people are searching on the web and correlating
that data with other sources like hospital records.

We have grown so accustomed to the tight level of integration between different
services and information sources, that behaviours similar to those presented
above are nowadays expected. User requirements have heightened, and products
can fail to gain traction if they don’t exploit the data that is available
around us in new and innovative ways; different computing devices and services
must discover themselves and make mutual use of the information that they
produce or consume, meshing together in what is called a Pervasive System.

Firstly envisioned by Mark Weiser \cite{weiser1991computer}, Pervasive Systems
are connected networks of independent and heterogeneous devices, whose ultimate
goal is to assist people in a way that is effectively invisible to the final
user. They are the result of a post-pc era, where scores of computing gadgets
are disseminated in our surroundings, enhancing our capabilities to sense the
world and providing ubiquitous access to information.

From a software and hardware perspective, a Pervasive Systems is a rich and
varied environment, hosting myriads of different network protocols, data
formats, and incompatible cpu architectures. Tapping the ever increasing stream
of data produced in such a heterogeneous context, and using it to build
advanced and connected products, can easily become a daunting task. Since
Weiser’s seminal paper, several endeavours have explored different techniques
for simplifying the task of building and designing Pervasive Systems, many of
    which stemmed from the research on Wireless Sensor Networks (WSN).


\section{Wireless Sensor Networks and beyond}

As suggested by the name, Wireless Sensor Networks (WSNs) are networks of
wirelessly connected devices called nodes or motes, that are able to measure or
detect physical properties from their surrounding environment. Although the
original acronym only references the sensory features of such systems, current
usage of the term WSN is commonly extended to include devices with actuation
capabilities.

Wireless Sensor Networks provide a low-cost and effective solution for
monitoring physical phenomena: several inexpensive nodes may be scattered
around the area of observation, without requiring an explicit configuration or
a wired communication infrastructure. Data is autonomously routed from the
point of origin to the interested consumers, where it is usually aggregated,
analysed and presented to the user. Flexibility and ease of deployment make
WSNs the ideal candidate for a plethora of applications, covering home
automation, theft prevention systems, healthcare, control of environmental
hazards and monitoring of production lines.

The rapid increase in popularity of WSNs, coupled with the intrinsic
difficulties of working in a variegated software and hardware environment,
spurred the development of a wide variety of frameworks, middlewares, and
ad-hoc programming languages designed for providing an easy way of access to
the functionalities offered by networks of sensing and actuating devices and
Pervasive Systems alike.

This thesis describes the design and implementation of one of these software
systems, an asynchronous data access middleware for Pervasive Networks dubbed
\textit{New PerLa Middleware}. Its history, as will be shown in
chapter~\ref{cha:perlasystem}, is strongly intertwined with the PerLa Query
Language \cite{tse_perla}, a declarative SQL-like language for collecting data
from WSNs and other distributed sensing networks. However, before delving into
the substance of this middleware, the next section will provide a short survey
on the different approaches proposed in literature for the management of
Pervasive Systems.


\section{Data management in Pervasive Systems}

Early experiences in managing sensing networks were based on ad-hoc systems
that provided bespoken solutions to specific applications. These approaches
were usually built using proprietary hardware and highly customized software
architectures, whose design was ultimately concerned with the implementation of
a limited and well-defined series of task-oriented requirements. As shown in
\cite{hartung2006firewxnet} \cite{mainwaring2002wireless}
\cite{werner2005monitoring} \cite{juang2002energy} these highly personalized
systems are poorly reuseable, since their architecture usually fails to provide
a clear separation of concerns between the pure data access mechanisms employed
to control the underlying sensing network, and the particular behaviour
required by the specific application. It soon became clear that this strategy
could not become a viable model for a simple and effective development of
pervasive applications, as all the effort and expertise put in the
implementation of such systems could not be efficiently exploited and shared in
new projects.

TinyDB \cite{madden2005tinydb} is one of the first endeavours that tried to
provide a generic abstraction suitable for the development of sensing
applications based on Pervasive Systems. A WSN managed by TinyDB is in fact
presented to the final users as a vast streaming database controlled through
SQL-like queries, which are appropriately interpreted and distributed to the
single nodes of a pervasive system in order to specify a globally coordinated
behaviour. Despite the general high-level abstraction employed by this system,
the execution of TinyDB queries requires all nodes in the sensing network to be
running TinyOS \cite{levis2005tinyos}, an embedded operating system
specifically designed to be used in sensing devices, therefore limiting the
variety of devices which can be managed using this approach.

Similarly to TinyDB, DSN \cite{chu2006entirely} aims at providing a
database-like abstraction of an entire Pervasive System, which can be queried
and controlled using a Datalog dialect dubbed Snlog. A comparable ``WSN as a
database'' abstraction, as will be shown in chapter~\ref{cha:perlasystem}, is a
characteristic feature of the PerLa System as well.

Other projects tried to approach the problem of handling a Pervasive System by
shifting the focus on the management of highly heterogeneous networks of
sensing devices. SWORD \cite{sword} tries to achieve this goal by providing a
central infrastructure that can be used to monitor events and signals collected
from a distributed mesh of nodes. Unfortunately, its XML-over-HTTP messaging
system considerably reduces the variety of devices that can be integrated. GSN
\cite{aberer2006global}, on the other hand, is a Java Middleware based on the
concept of \textit{Virtual Sensor}, a high-level software component aimed at
providing a common interface that can be used to collect information generated
by a single device of a sensing network. Virtual Sensors are created by means
of a declarative XML descriptor, and can be interfaced with the remote
endpoints through a specific device driver dubbed \textit{wrapper}.
Unfortunately GSN does not provide any mechanism for automating the deployment
of new Virtual Sensors.

The TinyREST \cite{luckenbach2005tinyrest} project attempts to provide a
RESTful \cite{fielding2000architectural} data access interface to each node
available in a Pervasive System. This goal is achieved through a server
infrastructure that performs a mapping between HTTP methods and various
Input/Output operations performed by the devices of a sensing network. As a
result, an HTTP GET request can be used to sample a physical phenomena, whereas
a PUT operation allows users to command actuators or set variables on remote
devices. Its reliance on the TinyOS makes it vulnerable to the same critiques
made to TinyDB.

Contiki \cite{dunkels2004contiki} is a lightweight and portable operating
system specifically designed for memory-limited devices. Differently from
TinyOS, Contiki tries to foster the same programming model available to desktop
machines. Its kernel, whose footprint may ary between 10 and 30 KB of RAM,
provides in fact a fully functional IPv4 and IPv6 networking stack, a
multi-thread concurrency mechanism based on protothreads and an over-the-air
programming mechanism. Differently from the other projects discussed in this
section, Contiki does not provide any high-level abstraction for accessing the
information produced by a Pervasive System.
