\chapter{Complete XML Device Descriptor examples}

\section{Example 1}

The Device Descriptor portrayed in this first example is employed to create a
Simulator \texttt{FPC}, a useful tool that helps testing the PerLa Middleware
even when no physical sensor nodes are available. This simulated device exposes
two attributes: a read-only float representing the temperature in Celsius
degrees (\texttt{temp\_c}), and a write-only integer that will be used to set
the sampling period (\texttt{period}).

This \texttt{FPC} contains one \texttt{Channel} of type
\texttt{SimulatorChannel}, which generates new data samples without requiring a
connection to a real sensing device. In our example, the
\texttt{SimulatorChannel} is configured with a single data generation routine,
named \texttt{temperature}, that automatically creates new data samples in
increments of 0.1\degree, spanning from a minimum of 16\degree~C to a maximum
of 20\degree~C. The \lstinline!<request>! section shows that a single
\texttt{IORequest} object is enough to drive the single data generator
available in this \texttt{FPC}.

Only two \texttt{Message} types are declared in the Device Descriptor: a
\texttt{temperature-msg} message, which contains a single field of type
\texttt{float}, and a \texttt{sampling-period} message, composed of only an
\texttt{integer} field. These two messages will be used to collect temperature
samples and to set the desired sampling period in the \texttt{SimulatorChannel}
respectively.

As can be seen from the \lstinline!<operation>! section of this descriptor, the
simulator device exposes a periodic sampling operation named
\texttt{temp-periodic}. It is important to note that the \texttt{start Script}
initializes the \texttt{SimulatorChannel} according to the sampling rate
specified by the user, which is retrieved with a \lstinline!${arg['period']}!
expression. New output records are create by the \texttt{on Script} upon
arrival of each \texttt{temperature-msg}. Finally, the generation of new data
samples can be stopped by setting the sampling frequency to zero.

~\\
\lstset{language=XML}
\begin{lstlisting}
<?xml version="1.0" encoding="UTF-8"?>
<device type="Weather simulator"
    xmlns="http://perla.dei.org/device"
    xmlns:i="http://perla.dei.org/device/instructions"
    xmlns:sim="http://perla.dei.org/channel/simulator">

  <attributes>
    <attribute id="temp_c" type="float" permission="read-only"/>
    <attribute id="period" type="integer" permission="write-only"/>
  </attributes>

  <channels>
    <sim:channel id="simulator">
      <sim:generator id="temperature">
        <sim:field name="temperature" strategy="step"
                   type="float" min="16" max="20" increment="0.1"/>
        </sim:generator>
    </sim:channel>
  </channels>
  
  <messages>
    <sim:message id="temperature-msg">
      <sim:field name="temperature" type="float"/>
    </sim:message>
    <sim:message id="sampling-period">
      <sim:field name="period" type="integer"/>
    </sim:message>
  </messages>
  
  <requests>
    <sim:request id="temperature-request" generator="temperature"/>
  </requests>
  
  <operations>
    <periodic id="temp-periodic">
      <start>
        <i:create variable="period" type="sampling-period"/>
        <i:set variable="period" field="period" value="${arg['period']}"/>
        <i:submit request="temperature-request" channel="simulator">
          <i:param name="period" variable="period"/>
        </i:submit>
      </start>
      <stop>
        <i:create variable="period" type="sampling-period"/>
        <i:set variable="period" field="period" value="0"/>
        <i:submit request="temperature-request" channel="simulator">
          <i:param name="period" variable="period"/>
        </i:submit>
      </stop>
      <on message="temperature-msg" variable="result">
        <i:put expression="\${result.temperature}" attribute="temp_c" />
        <i:emit />
      </on>
    </periodic>
  </operations>

</device>
\end{lstlisting}

\section{Example 2}

This second Device Descriptor is used to configure an \texttt{FPC} for
collecting data from a RESTful meteorological service. The remote API is
accessed by means of a \texttt{HTTPChannel}, which retrieves a complete
JSON-encoded weather report for the city of Milan.

The service chosen for this application returns its results using a fairly
complex JSON entity, whose structure has been replicated in the
\lstinline!<message>! section of this descriptor. The main \texttt{weather}
object is in fact a wrapper for three other components: a \texttt{coord}
message, a \texttt{data} message, and a \texttt{wind} message. It is worth
noting that these three sub-objects are constructed using only primitive
fields, and that they are not intended to be used on their own; their only
purpose is to be included as the constituent elements of a \texttt{weather}
message.

The stateless nature of the HTTP protocol is underscored by the simplicity of
the \texttt{HTTPChannel} declaration, which doesn't require any parameter
except for its own identifier, and by the related \texttt{HTTPIORequest}, where
all information required for querying the RESTful API endpoint is stored. These
two \texttt{FPC} components are then employed in the \lstinline!<get>!
\texttt{Operation}, which is tasked with creating a new record from the
\texttt{weather} object retrieved from the web service. 

~\\
\lstset{language=XML}
\begin{lstlisting}
<?xml version="1.0" encoding="UTF-8"?>
<device type="REST Weather station"
    xmlns="http://perla.dei.org/device"
    xmlns:js="http://perla.dei.org/fpc/message/json"
    xmlns:http="http://perla.dei.org/channel/http"
    xmlns:i="http://perla.dei.org/device/instructions">

  <attributes>
    <attribute id="city" type="string" permission="read-only"/>
    <attribute id="temp_k" type="float" permission="read-only"/>
    <attribute id="temp_c" type="float" permission="read-only"/>
    <attribute id="temp_f" type="float" permission="read-only"/>
    <attribute id="pressure" type="float" permission="read-only"/>
    <attribute id="humidity" type="float" permission="read-only"/>
    <attribute id="wind_speed" type="float" permission="read-only"/>
    <attribute id="wind_deg" type="float" permission="read-only"/>
  </attributes>

  <channels>
    <http:channel id="http"/>
  </channels>

  <messages>
    <js:object id="coord">
      <js:value name="lon" type="string"/>
      <js:value name="lat" type="string"/>
    </js:object>

    <js:object id="data">
      <js:value name="temp" type="float"/>
      <js:value name="pressure" type="float"/>
      <js:value name="humidity" type="float"/>
      <js:value name="temp_min" type="float"/>
      <js:value name="temp_max" type="float"/>
    </js:object>

    <js:object id="wind">
      <js:value name="speed" type="float"/>
      <js:value name="deg" type="float"/>
    </js:object>

    <js:object id="weather">
      <js:value name="coord" type="coord"/>
      <js:value name="data" type="data"/>
      <js:value name="wind" type="wind"/>
    </js:object>
  </messages>

  <requests>
    <http:request id="weather-mi"
                  host="http://api.openweathermap.org/data/2.5/weather?q=Milan,it"
                  method="get" />
  </requests>

  <operations>
    <get id="weather-mi">
      <i:submit request="weather-mi" channel="http"
                variable="result" type="weather"/>
      <i:put expression="Milan" attribute="city"/>
      <i:put expression="${result.data.temp}" attribute="temp_k"/>
      <i:put expression="${result.data.temp - 272.15}" attribute="temp_c"/>
      <i:put expression="${(result.data.temp - 273.15) * 9 / 5 + 32}"
             attribute="temp_f"/>
      <i:put expression="${result.data.pressure}" attribute="pressure"/>
      <i:put expression="${result.data.humidity}" attribute="humidity"/>
      <i:put expression="${result.wind.speed}" attribute="wind_speed"/>
      <i:put expression="${result.wind.deg}" attribute="wind_deg"/>
      <i:emit/>
    </get>
  </operations>

</device>
\end{lstlisting}

\section{Example 3}

A purely educational Device Descriptor designed to showcase the following PerLa
Middleware features:

\begin{itemize}
    \item read-write, read-only write-only and static attributes;
    \item \texttt{SimulatorChannel} value generator configurations for all
        available data types;
    \item \texttt{Get} and \texttt{Set Operations};
    \item \texttt{Periodic Operations} with single and multiple
        \lstinline!<on>! \texttt{Scripts}.
\end{itemize}

~\\
\lstset{language=XML}
\begin{lstlisting}
<?xml version="1.0" encoding="UTF-8"?>
<device type="test"
    xmlns="http://perla.dei.org/device"
    xmlns:i="http://perla.dei.org/device/instructions"
    xmlns:sim="http://perla.dei.org/channel/simulator">

  <attributes>
    <attribute id="integer" type="integer" permission="read-write"/>
    <attribute id="float" type="float" permission="read-write"/>
    <attribute id="boolean" type="boolean" permission="read-write"/>
    <attribute id="string" type="string" permission="read-write"/>
    <attribute id="event" type="boolean" permission="read-only"/>
    <attribute id="period" type="integer" permission="write-only"/>
    <attribute id="static" type="integer" access="static" value="5"/>
  </attributes>

  <channels>
    <sim:channel id="simulator">
      <sim:generator id="all">
        <sim:field name="type" strategy="static" value="all"/>
        <sim:field name="integer" strategy="dynamic"
                   type="integer" min="12" max="32"/>
        <sim:field name="float" strategy="dynamic"
                   type="float" min="450" max="600"/>
        <sim:field name="string" strategy="dynamic"
                   type="string" min="10" max="15"/>
      </sim:generator>
      <sim:generator id="integer">
        <sim:field name="type" strategy="static" value="integer"/>
        <sim:field name="integer" strategy="dynamic"
                   type="integer" min="47" max="58"/>
      </sim:generator>
      <sim:generator id="string">
          <sim:field name="type" strategy="static" value="integer"/>
          <sim:field name="string" strategy="dynamic"
                     type="string" min="5" max="5"/>
      </sim:generator>
      <sim:generator id="boolean">
        <sim:field name="type" strategy="static" value="boolean"/>
        <sim:field name="boolean" strategy="dynamic" type="boolean"/>
      </sim:generator>
      <sim:generator id="event">
        <sim:field name="type" strategy="static" value="event"/>
        <sim:field name="event" strategy="dynamic" type="boolean"/>
      </sim:generator>
    </sim:channel>
  </channels>

  <messages>
    <sim:message id="set-msg">
      <sim:field name="test" type="integer"/>
    </sim:message>
    <sim:message id="sampling-period">
      <sim:field name="period" type="integer"/>
    </sim:message>
    <sim:message id="all-msg">
      <sim:field name="type" type="string" qualifier="static" value="all"/>
      <sim:field name="integer" type="integer"/>
      <sim:field name="float" type="float"/>
      <sim:field name="string" type="string"/>
    </sim:message>
    <sim:message id="integer-msg">
      <sim:field name="type" type="string" qualifier="static" value="integer"/>
      <sim:field name="integer" type="integer"/>
    </sim:message>
    <sim:message id="string-msg">
      <sim:field name="type" type="string" qualifier="static" value="string"/>
      <sim:field name="string" type="string"/>
    </sim:message>
    <sim:message id="boolean-msg">
      <sim:field name="type" type="string" qualifier="static" value="boolean"/>
      <sim:field name="boolean" type="boolean"/>
    </sim:message>
    <sim:message id="event-msg">
      <sim:field name="type" type="string" qualifier="static" value="event"/>
      <sim:field name="event" type="boolean"/>
    </sim:message>
  </messages>

  <requests>
    <sim:request id="all-request" generator="all"/>
    <sim:request id="integer-request" generator="integer"/>
    <sim:request id="string-request" generator="string"/>
    <sim:request id="boolean-request" generator="boolean"/>
    <sim:request id="event-request" generator="event"/>
  </requests>

  <operations>
    <get id="integer-get">
      <i:submit request="integer-request"
                channel="simulator" variable="result" type="integer-msg"/>
      <i:put expression="${result.integer}" attribute="integer"/>
      <i:emit/>
    </get>
    <get id="string-get">
      <i:submit request="string-request"
                channel="simulator" variable="result" type="string-msg"/>
      <i:put expression="${result.string}" attribute="string"/>
      <i:emit/>
    </get>
    <set id="integer-set">
      <!-- Just for test purposes -->
      <i:create variable="set-data" type="set-msg"/>
      <i:set variable="set-data" field="test" value="${param['integer']}"/>
    </set>
    <periodic id="all-periodic">
      <start>
        <i:create variable="period" type="sampling-period"/>
        <i:set variable="period" field="period" value="${param['period']}"/>
        <i:submit request="all-request" channel="simulator">
          <i:param name="period" variable="period"/>
        </i:submit>
      </start>
      <stop>
        <i:create variable="period" type="sampling-period"/>
        <i:set variable="period" field="period" value="0"/>
        <i:submit request="all-request" channel="simulator">
          <i:param name="period" variable="period"/>
        </i:submit>
      </stop>
      <on message="all-msg" variable="result">
        <i:put expression="${result.integer}" attribute="integer" />
        <i:put expression="${result.float}" attribute="float" />
        <i:put expression="${result.string}" attribute="string" />
        <i:emit />
      </on>
    </periodic>
    <periodic id="multiple-periodic">
      <start>
        <i:create variable="period" type="sampling-period"/>
        <i:set variable="period" field="period" value="${param['period']}"/>
        <i:submit request="integer-request" channel="simulator">
          <i:param name="period" variable="period"/>
        </i:submit>
        <i:submit request="boolean-request" channel="simulator">
          <i:param name="period" variable="period"/>
        </i:submit>
      </start>
      <stop>
        <i:create variable="period" type="sampling-period"/>
        <i:set variable="period" field="period" value="0"/>
        <i:submit request="integer-request" channel="simulator">
          <i:param name="period" variable="period"/>
        </i:submit>
        <i:submit request="boolean-request" channel="simulator">
          <i:param name="period" variable="period"/>
        </i:submit>
      </stop>
      <on message="integer-msg" variable="result" sync="true">
        <i:put expression="${result.integer}" attribute="integer" />
        <i:emit />
      </on>
      <on message="boolean-msg" variable="result">
        <i:put expression="${result.boolean}" attribute="boolean" />
        <i:emit />
      </on>
    </periodic>
    <async id="event-async">
      <start>
        <i:create variable="period" type="sampling-period"/>
        <i:set variable="period" field="period" value="200"/>
        <i:submit request="event-request" channel="simulator">
          <i:param name="period" variable="period"/>
        </i:submit>
      </start>
      <on message="event-msg" variable="result">
        <i:put expression="${result.event}" attribute="event"/>
        <i:emit />
      </on>
    </async>
  </operations>

</device>
\end{lstlisting}

